%

% ---------------------------------------------------------------------------------------------------------------------------------
% -- /home/yarahmadi/work/ConvexOptim/2017/alle/activities/papers/1D_Perspective/Michael_Needs --
% ---------------------------------------------------------------------------------------------------------------------------------

\documentclass[a4paper,11pt]{article}
% python code
\usepackage{listings}

% graphics
\usepackage{tikz}
\usepackage{pgfplots}
\pgfplotsset{width=4.5cm,compat=1.12}
\usepgfplotslibrary{fillbetween}

\usepackage[utf8]{inputenc}
\usepackage[T1]{fontenc}

\usepackage{graphicx,amssymb,amsmath}
\usepackage{epsfig,float}
\usepackage{pstricks,pst-node,pst-text,pst-3d}
%\usepackage{showlabels} %---> showing labels
%\usepackage{color} % ---> text in color
\usepackage{xfrac} % ---> / fraction
\usepackage{mathtools} % \coloneqq
\usepackage{algorithmic} %---> algorithm
\usepackage{algorithm}
\usepackage{booktabs}
\allowdisplaybreaks % break formulas across pages
\usepackage{cancel} % cancel terms
\usepackage{ulem} % cancel terms

\usepackage{mathtools}
\usepackage{tikz,tikz-3dplot}
\usepackage{pgfplots}
\usepackage[ngerman]{babel}  

\usepackage{graphics}
\usepackage{tikz}
\pgfplotsset{compat=1.10}
\usepgfplotslibrary{fillbetween}

\usepackage{fontawesome} % light symbol in figure
\usepackage{enumitem}
\usepackage{mathtools}
\usepackage{xfrac}
\newtheorem{lema}{Lemma}[subsection] 
\newtheorem{theo}{Theorem}[subsection] 
\newtheorem{prop}{Proposition}[subsection] 
\newtheorem{defin}{Definition}[subsection]
\newtheorem{examp}{Example}[subsection] 
\newtheorem{coroll}{Corollary}[subsection] 
\newtheorem{proof}{Proof}[subsection] 
\usepackage[page,toc]{appendix}
\allowdisplaybreaks %allow big equations to break over pages.
%--
\usepackage{color, colortbl} %table probability
\definecolor{Gray}{gray}{0.9}
\definecolor{LightCyan}{rgb}{0.88,1,1}
\usepackage[first=0,last=9]{lcg}
\newcommand{\ra}{\rand0.\arabic{rand}}
%---

%--breal aligned text (normal vector)
\usepackage{amsmath}     
\newcommand{\dd}[1]{\mathrm{d}#1}      
\allowdisplaybreaks 
%---------------
\begin{document}
%--------------- 
\title{
Support vector machine$\colon$A quick look on cost function
}
\maketitle
%-----------------------------------
\section{Hinge Loss Function}\label{sec-1}
%-----------------------------------
It is written as
%
\begin{equation}
	\mathcal{H}\left(x,y,f\left(x\right)\right)=
	\begin{cases}
       0                                 , & \text{if }         y f\left(x\right) \geq 1\\
       1-y f\left(x\right)          , & \text{otherwise}
	\end{cases}
\label{eq:1}
\end{equation}
%
with $x\in\mathbb{R}^n$ and $y\in\left\{-1,+1\right\}$ to be the training samples and their corresponding labels. Here, $f\left(x\right)\coloneqq\left<x\cdot w\right>\in\left\{-1,+1\right\}$ is the predicted label of the training sample $x$. Here, $w\in\mathbb{R}^n$, with $n$ and $m$ defined in $\mathbb{N}$.
%-----------------------------------
\section{Objective Function}\label{sec-2}
%-----------------------------------
Based on~\eqref{eq:1}, support vector machine (SVM) objective's function is written as
%
\begin{equation}
\min_{w}{\left(\lambda\left\|w\right\|^2+\sum_{i=1}^{n}{\mathcal{H}
         \left(x,y,\overbrace{f\left(x\right)}^{\coloneqq\left<x\cdot w\right>}\right)}\right)}
\label{eq:2}         
\end{equation}
%
with $\lambda\in\left(0,+\infty\right)$.
%----------------------------
\section{Optimization}\label{sec-3}
%----------------------------
To optimize~\eqref{eq:2}, we need its derivatives
%
\begin{equation}
\frac{\partial}{\partial w}
\left(
\min_{w}{
           \left(
           \lambda\left\|w\right\|^2 +
           \sum_{i=1}^{n}
           {
           \mathcal{H}
           \left(x,y,f\left(x\right)\right)
           }
           \right)
        }
\right)
\label{eq:3}
\end{equation}
%
that can be taken separately as 
%
\begin{equation}
\frac{\partial}{\partial w_{k}}
\left(
	\lambda\left\|w\right\|^2
\right)= 2\lambda w_{k}
\label{eq:4}
\end{equation}
%
and
%
\begin{equation}
\frac{\partial}{\partial w_{k}}
\left(
	\mathcal{H}\left(x,y,f\left(x\right)\right)
\right)
= 
\begin{cases}
   0 , & \text{if }  y_i \left<x_i\cdot w_{k}\right> \geq 1\\
   -y_i x_i , & \text{otherwise}
\end{cases}
\label{eq:5}
\end{equation}
%
with $k\in\left\{1,\cdots,n\right\}$ leading us to the below stochastic gradient descent updation rules
%
\begin{equation}
w_k=w_k - \eta \left( 2\lambda w_k - \left<x_i\cdot w_{k}\right> \right)
\label{eq:6}
\end{equation}
%
if $y_i \left<x_i\cdot w_{k}\right> < 1$ and otherwise
%
\begin{equation}
w_k=w_k - \eta \left( 2\lambda w_k \right).
\label{eq:7}
\end{equation}
%
One writes the snippet python code corresponding to~\eqref{eq:6} and~\eqref{eq:7} as
%
\begin{lstlisting}[language=Python]
    
def SVM_SGD(X,Y):
  w = np.zeros(len(X[0]))
  eta = 1
  epochs = 100000
  for epoch in range(1,n):
   for i, x in enumerate(X):
    if (Y[i]*np.dot(X[i], w)) < 1:
	 w = w + eta * ( (X[i] * Y[i]) + (-2  *(1/epoch)* w) )
	else:
	 w = w + eta * (-2  *(1/epoch)* w)
return w
\end{lstlisting}
%
\end{document}
